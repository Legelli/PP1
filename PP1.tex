% !TEX TS-program = XeLaTeX
%!TEX encoding = UTF-8 Unicode
%==================================================
%      PREAMBOLO e DICHIARAZIONI INIZIALI
%==================================================
\documentclass[10pt,oneside,a4paper]{article}

\usepackage[utf8]{inputenc} 
\usepackage[italian]{babel}
\usepackage[T1]{fontenc}
\usepackage{siunitx} %Inserisce automaticamente i dati con le unità  di misura correttamente formattate del SI (utilizzo: \SI{0.82}{m^2}, in generale \SI{misura con il punto decimale}{unità  di misura})
\sisetup{output-decimal-marker = {.}, separate-uncertainty = true, input-uncertainty-signs = \pm, detect-weight=true, detect-family=true} %per usare SI con il punto decimale
\usepackage{listings} %Per citare codice informatico formattandolo correttamente
\usepackage{amsmath,amsthm,verbatim,amssymb,amsfonts,amscd,graphicx,mathtools}
\usepackage[makeroom]{cancel}
\newcommand{\abs}[1]{\left\lvert\,#1\,\right\rvert}
\usepackage{geometry}
\usepackage{epigraph}
\usepackage{booktabs}	%tabelle migliorate
\usepackage{tablefootnote}	%note a piè di pagina in tabella
\usepackage{threeparttable} %tabella con note a piè di tabella
\usepackage{caption}	%descrizione per figure
\usepackage{dblfnote}
\captionsetup{tableposition=top,figureposition=bottom,font=small} %setup descrizione
\usepackage{float}
\usepackage{esvect} %vettori
\usepackage{longtable} %tabelle lunghe
\usepackage[dvipsnames]{xcolor}
\definecolor{sepia}{HTML}{80002A}
\usepackage[colorlinks=true, citecolor=black, linkcolor=sepia, urlcolor=black]{hyperref}
\usepackage{mathrsfs}
\usepackage{circuitikz}
\tikzset{
  font={\fontsize{7pt}{12}\selectfont}}
\ctikzset{bipoles/resistor/height=0.2}
\ctikzset{bipoles/resistor/width=0.4}
\ctikzset{bipoles/diode/height=0.3}
\ctikzset{bipoles/diode/width=0.3}
\ctikzset{tripoles/american nand port/height=0.7}
\ctikzset{tripoles/american nand port/width=0.8}
\usepackage{enumitem} %Liste senza spazi verticali
\setlist{noitemsep}
\usepackage{amsmath}
\usepackage{hyperref}
\usepackage{pst-optexp} %Diagrammi ottici
\usepackage{physics} %Ambienti utili


\interfootnotelinepenalty=10000


\usepackage{multicol}
\newenvironment{Figure}
  {\par\medskip\noindent\minipage{\linewidth}}
  {\endminipage\par\medskip}

%\newcommand{\var}{\operatorname{var}}
%\newcommand{\cov}{\operatorname{cov}}


\usepackage{listings} %Per inserire codice
\lstdefinestyle{CStyle}{
    backgroundcolor=\color{backgroundColour},   
    commentstyle=\color{mGreen},
    keywordstyle=\color{magenta},
    numberstyle=\tiny\color{mGray},
    stringstyle=\color{mPurple},
    basicstyle=\footnotesize\ttfamily,
    breakatwhitespace=false,         
    breaklines=true,                 
    captionpos=b,                    
    keepspaces=true,                 
    numbers=left,                    
    numbersep=5pt,                  
    showspaces=false,                
    showstringspaces=false,
    showtabs=false,                  
    tabsize=2,
    language=C
}

\definecolor{color1}{RGB}{90,0,0} % Color of the article title and sections
\definecolor{color2}{RGB}{0,20,50} % Color of the boxes behind the abstract and headings
\definecolor{mGreen}{rgb}{0,0.6,0}
\definecolor{mGray}{rgb}{0.5,0.5,0.5}
\definecolor{mPurple}{rgb}{0.58,0,0.82}
\definecolor{backgroundColour}{rgb}{0.95,0.95,0.92}


%==================================================
%                  PRIMA PAGINA
%==================================================

\title{\textsc{\textbf{Esperienza 1}: Legge di Malus -- Angolo di Brewster}}
\author{\small{G. Galbato Muscio} \and \small{F. Ghimenti} \and \small{L. Gravina} \and \small{L. Graziotto}}
\date{28 Marzo 2019}

\begin{document}
	\begin{figure}
		\centering
		\includegraphics[scale=0.5, trim={2.8cm 8.9cm 0 9cm}, clip]{logo.png}
	\end{figure}
	\maketitle
	\begin{center} 
		\fbox{{\fontsize{12pt}{8mm}\textsc{Gruppo D1-1}}} \\
	\end{center}
\hrule
\vspace{0.5cm}
\renewcommand{\abstractname}{Abstract}
\begin{abstract}
Si misura l'intensità di un laser HeNe\footnote{\url{https://www.dropbox.com/s/5aqzs2uykfi8lms/8-Coherence_function_He_Ne.pdf?dl=0}} attraverso un filtro polaroid al variare dell'angolo relativo tra asse principale e direzione di polarizzazione della luce per verificare la legge di Malus. Si studia l'andamento dell'intensità delle componenti TE e TM della componente riflessa da un dielettrico (perspex) al variare dell'angolo di incidenza, si misura l'angolo di Brewster.
\end{abstract}
\vspace{4cm}
\tableofcontents %Indice
\newpage


\pagebreak


\begin{multicols}{2}
%==================================================
%             APPARATO STRUMENTALE
%==================================================
\section{Apparato strumentale}

\begin{pspicture}(2.5,4)
                                 \psset{beam}
                                 % wrong, also beam width is changed
                                 \mirror[linewidth=3\pslinewidth](0,3)(2,3)(2,2)
                                 % correct result
                                 \addtopsstyle{OptComp}{linewidth=3\pslinewidth}
                                 \mirror(0,1)(2,1)(2,0)
                               \end{pspicture}

%ESEMPIO DI CODICE
%\begin{lstlisting}[style=CStyle, caption={Codice per la misura dei tempi di esecuzione della moltiplicazione}, label=lst:moltiplicazione]
%#define MAX 10000
%
%int p, q;
%long unsigned t0, t1;
%
%void setup() {
%	Serial.begin(9600);
%	p=10; 
%}
%
%void loop() { 
%	Serial.print(p);
%	t0 = micros();
%	q = p*10;
%	t1 = micros();
%	p = p+1;	
%	Serial.print(p);
%	Serial.print(" ");
%	Serial.println(t1-t0); //stampa il tempo
%	while (p > MAX) { }; //il programma si arresta quando il numero supera MAX
%	}
%\end{lstlisting}


%ESEMPIO DI FIGURA
%\begin{Figure}
%	\begin{center}
%	\includegraphics[width=\linewidth]{comune.png}
%	\captionof{figure}{Istantanea dell'oscilloscopio per l'amplificatore differenziale, misura di $A_c$}
%	\label{fig:Ac_differenziale}
%	\end{center}
%\end{Figure}


%ESEMPIO DI TABELLA
%\begin{center}
%\captionof{table}{Misure per la stima di $A_c$}
%\label{tab:Ac_differenziale}
%\begin{tabular}{c|c|c|c}
%$f$ [\SI{}{Hz}] & $V_i$ [\SI{}{V}] & $v_o$ [\SI{}{mV}] & $A_c = v_o / V_i$ \\
%\hline
%      149.5 &        3.90 &         11.3 & 2.90e-03 \\
%      222.0 &        3.90 &         11.5 & 2.95e-03 \\
%      281.0 &        3.90 &         11.8 & 3.03e-03 \\
%      359.0 &        3.90 &         11.8 & 3.03e-03 \\
%      461.0 &        3.90 &         12.2 & 3.13e-03 \\
%\hline
%\end{tabular}
%\end{center}


\end{multicols}


\end{document}